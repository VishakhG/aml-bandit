\section{Methodology}

\textbf{Feature extraction:}
The dataset we use for the experiments discussed in this document is composed of features extracted from the MIMIC database.
More sepcifically rows the dataset correspond to individual admissions and columns are features.
The rows are all represented as one-hot-encodings where an index takes the value 1 if the feature at that index was recorded for the admission.
The total set of features were taken from  unique prescriptions, procedures and diagnoses events from the MIMIC.
We did not actually use the codes verbatim, but found UMLS concepts that most closely matched each code. Unified Medical Language System (UMLS)
concepts are a hierarchical categorization of medical concepts which seved to combine icd-9 codes associated with similar medical concepts, and to 
put drug codes and ICD-9 codes in the same space. Since the number of features at this point was $\approx$ 20,000, we removed the rarest features to 
trim it down to $\approx 800$, with $\approx$ 50,000 admissions in total. We then split the dataset into train, validation and test subsets which have 
sizes observing the numbers in \ref{table:splits}.


\begin{table}[h!]
\centering
\caption{Train-Test-Validation}
\label{table:splits}
\begin{tabular}{|l|l|l|l|}
\hline
               & Mortality & Aspirin & Cancer \\ \hline
Examples in train      & 41282     & 2037    & 734    \\ \hline
Examples in validation & 5899      & 623     & 231    \\ \hline
Examples in test       & 11795     & 304     & 111       \\ \hline
\end{tabular}
\caption*{Sizes of the train/test/validation splits for different labels.}
\end{table}