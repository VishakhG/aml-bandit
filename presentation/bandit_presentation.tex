\documentclass{beamer}
\usepackage{algorithm}
\usepackage{algorithmic}
\usepackage{listings}

\DeclareMathOperator*{\argmin}{arg\,min}

\usepackage{graphicx}

\usetheme{metropolis}          
\title{Bandit Problems}
\date{\today}
\author{Vishakh Gopu, Frederik Jensen}
\institute{Advanced machine learning}
\begin{document}
\maketitle

%%%%%%%%%%%%%%%%%%%%%%%%%%%%%%%%%%%%%%%%%%%%%%%%%%%%%%% 
% Introduction
%%%%%%%%%%%%%%%%%%%%%%%%%%%%%%%%%%%%%%%%%%%%%%%%%%%%%%% 

\section{Introduction}
\begin{frame}{Overview of papers}   
  Three main problem settings: 
  \begin{enumerate}
  \item
    Restless bandits
    \begin{itemize}
    \item
      bertsimas, Dimitris,Nin ̃o-Mora, Jose ́. 2000. Rest- less Bandits, Linear Programming
      Relaxations, and a
      Primal-Dual Index Heuristic. Operations Research, 48(1), 80–90.
    \end{itemize}
  \item
    Graph structured feedback
    \begin{itemize}
    \item
      Alon, Noga, Cesa-Bianchi, Nicolo`, Gentile, Claudio, Mannor, Shie, Mansour, Yishay,
      Shamir, Ohad. 2014. Nonstochastic Multi-Armed Bandits with Graph- Structured Feedback.
      CoRR, abs/1409.8428.
    \end{itemize}
  \item
    Handling the variance in rewards
    
    \begin{itemize}
    \item
      Hazan, Elad, Kale, Satyen. 2009. Better Algorithms for Benign Bandits. Pages 38–47 of: 
      Proceedings of the Twentieth Annual ACM-SIAM Symposium on Discrete Algorithms. SODA ’09.
      Philadelphia, PA, USA: Society for Industrial and Applied Mathematics.
    \end{itemize}
  \end{enumerate}
\end{frame}

\begin{frame}{Outline}
  \begin{enumerate}
  \item
    A quick look at restless bandits
    
  \item
    Graph structured feedback and benign bandits in more depth
    \begin{itemize}
    \item
      Problem description
    \item
      Selected results
    \end{itemize}
    
  \item
    Interesting connections and relationships 
  \item
    Questions, exploration, future work
    
  \end{enumerate}
\end{frame}

\section{Restless bandits}
%%%%%%%%%%%%%%%%%%%%%%%%%%%%%%%%%%%%%%%%%%%%%%%%%%%%%%% 
% Restless bandits 
%%%%%%%%%%%%%%%%%%%%%%%%%%%%%%%%%%%%%%%%%%%%%%%%%%%%%%% 

\begin{frame}{Quick overview}
  % Talk about the problem setting and brief  high level description 
  % of the result (avoid showing any results)
  % Mention its in the interest of time
\end{frame}

\section{Non stochastic bandits with graph structured feedback}
%%%%%%%%%%%%%%%%%%%%%%%%%%%%%%%%%%%%%%%%%%%%%%%%%%%%%%% 
% Non stochastic bandits with graph structured feedback
%%%%%%%%%%%%%%%%%%%%%%%%%%%%%%%%%%%%%%%%%%%%%%%%%%%%%%% 
\begin{frame}{High level overview}

\end{frame} 

\begin{frame}{Problem formulation}

\end{frame}

\begin{frame}{KEY RESULT 1}
\end{frame}

\begin{frame}{KEY RESULT 2}

\end{frame}
\section{Better algorithms for benign bandits}
%%%%%%%%%%%%%%%%%%%%%%%%%%%%%%%%%%%%%%%%%%%%%%%%%%%%%%% 
% Total variation
%%%%%%%%%%%%%%%%%%%%%%%%%%%%%%%%%%%%%%%%%%%%%%%%%%%%%%% 
\begin{frame}{High level overview}

  \begin{enumerate}
  \item
    Losses might not be truly adversarial
    \begin{itemize}   
    \item
      Example: planning a route to work in traffic
    \end{itemize}
    \item
      Can we take advantage of low variation in our algorithm?
    \item
       Can we bound regret based on variability?
      \begin{itemize}
        \item
          Learning should depend on the \textit{total variation} of the losses
        \end{itemize}
  \end{enumerate}
\end{frame}

\begin{frame}{Problem description}
  \begin{enumerate}
    \item
    Online bandit convex optimization
    \begin{itemize}
      \item
        Adversarial but oblivious
      \item
        Point ${\bf x}_t$ from $\mathcal{K}$, compact set
      \item
        Cost vector $f_{t}$
      \end{itemize}
    \item
      Take the total variation
      $Q_T=\sum_{t=1}^T \| f_t - \mu \|^2$
    \item
      Recall barrier functions and self concordant functions
      \begin{itemize}
        \item
          Ensure that we always sample from the feasible set
        \end{itemize}
     \item
       FTRL ++
        \begin{align*}
          \argmin_{{\textbf{x}}_t\in\mathcal{K}}\eta \sum_{t=1}^{t-1} \tilde{f_{\tau}}^T + \mathcal{R}(x)
        \end{align*}

    \end{enumerate}
  \end{frame}
       
\begin{frame}[fragile]{Resevoir Sampling}
  \begin{itemize}
  \item
      Sample uniformly at random from a stream of unknown size
 \end{itemize}

  \lstset{language=Python}
  \lstset{frame=lines}
  %TODO cite jeremey kun
  \lstset{basicstyle=\footnotesize}
  \begin{lstlisting}
    import random

    def reservoirSample(stream):
        for k,x in enumerate(stream, start=1):
        if random.random() < 1.0 / k:
            chosen = x

    return chosen
  \end{lstlisting}
\end{frame}

\begin{frame}{Main algorithm}
  \begin{center}
    \scalebox{0.5}{
    \begin{minipage}{0.7\linewidth}
\begin{algorithm}[H]
  \begin{algorithmic}[1]

    \STATE{Input: $\eta > 0, \mathcal{V} \text{-self-concordant } \mathcal{R}, \text{ resevoir size parameter } k$} \\

    \STATE{
      Initialization: for all $i \in [n], j \in [k]$ set $S_{i,j} =0$ Set $x_i = argmin_{x\in \mathcal{K}}[\mathcal{R}(x)]$ and $\tilde{\mu_o}=0$ \\
      Let $\pi : \{1,2,...,nk\} \rightarrow \{1,2,..,nk\}$ be a random permutation.
    }\\

    \FOR{$t=1$  \TO T} 

    \STATE{
      Set $r=1$ with probability $\text{ min } \{ \frac{nk}{t}, 0 \}$, and $0$ with probability 
      $1 - \text{ min } \{\frac{nk}{t}, 1\}$
      \IF{r = 1} 

      \STATE{
        \IF{$t \leq nk$}
        \STATE{Set $i_t = (\pi(t) \text{ mod } n) + 1$}
        \ELSE
        \STATE{Set $i_t$ uniformly at random from $\{1,2,...,n\}$}
        \ENDIF \\
        
        Set $\tilde{\mu_t} \leftarrow SIMPLEXSAMPLE(i_t)$ \\
        Set $\tilde{f_t}=0$
      }
      
      \ELSE
      \STATE{
        Set $\tilde{\mu_t} = \mu_{t-1}$ \\
        Set $\tilde{f_t} \leftarrow ELLIPSOIDSAMPLE(x_t, \tilde{\mu_t})$
      }
      \ENDIF \\
      $x_{t+1} = argmin_{x \in \mathcal{K}}[\eta \sum_{\tau =1}^t \tilde{f_{\tau}^T} x + \mathcal{R}(x)]$
    }
    \ENDFOR
    

  \end{algorithmic}
  \caption{\label{alg:seeifrelin} Bandit online linear optimization}
\end{algorithm}
\end{minipage}
}
\end{center}
\end{frame}

\begin{frame}{Explore}
  \input{bbb_SIMPLEXSAMPLE}
\end{frame}
\begin{frame}{Explore-Exploit}
  \input{bbb_ELLIPSOIDSAMPLE}
\end{frame}


\begin{frame}{Bound}
  \begin{equation}
    E[Regret_t] = O(n \sqrt{\mathcal{V}Q log T} + n log^2(T) + n \mathcal{V}log(T))
  \end{equation}
\end{frame}

\section{Connections and exploration}
\begin{frame}{Connections between papers}
  \begin{enumerate}
  \item
    Letting novel mathematical tools guide the approach
  \item 
    Interpolating between problem settings
  \end{enumerate}
\end{frame}

\begin{frame}{Exploration}
  % talk about the idea we have.
\end{frame}

\section{Conclusion}
\begin{frame}{Open questions, future work}
  % Open questons from hazan, future ideas
\end{frame}

\section{Questions}
\end{document}