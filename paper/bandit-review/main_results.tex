\section{Overview of main results}

\subsection{Total variation}
In this section we go into the algorithms and techniques used in \citep{hazan}
in more detail to explore how they managed to bound regret by total variation in the
adversarial bandit setting.

\subsubsection{Algorithms}
First we state the algorithm developed in \citep{hazan} in full, and walk through them
to analyze the choices that were made.

%algorithms 
\begin{center}
    \scalebox{0.5}{
    \begin{minipage}{0.7\linewidth}
\begin{algorithm}[H]
  \begin{algorithmic}[1]

    \STATE{Input: $\eta > 0, \mathcal{V} \text{-self-concordant } \mathcal{R}, \text{ resevoir size parameter } k$} \\

    \STATE{
      Initialization: for all $i \in [n], j \in [k]$ set $S_{i,j} =0$ Set $x_i = argmin_{x\in \mathcal{K}}[\mathcal{R}(x)]$ and $\tilde{\mu_o}=0$ \\
      Let $\pi : \{1,2,...,nk\} \rightarrow \{1,2,..,nk\}$ be a random permutation.
    }\\

    \FOR{$t=1$  \TO T} 

    \STATE{
      Set $r=1$ with probability $\text{ min } \{ \frac{nk}{t}, 0 \}$, and $0$ with probability 
      $1 - \text{ min } \{\frac{nk}{t}, 1\}$
      \IF{r = 1} 

      \STATE{
        \IF{$t \leq nk$}
        \STATE{Set $i_t = (\pi(t) \text{ mod } n) + 1$}
        \ELSE
        \STATE{Set $i_t$ uniformly at random from $\{1,2,...,n\}$}
        \ENDIF \\
        
        Set $\tilde{\mu_t} \leftarrow SIMPLEXSAMPLE(i_t)$ \\
        Set $\tilde{f_t}=0$
      }
      
      \ELSE
      \STATE{
        Set $\tilde{\mu_t} = \mu_{t-1}$ \\
        Set $\tilde{f_t} \leftarrow ELLIPSOIDSAMPLE(x_t, \tilde{\mu_t})$
      }
      \ENDIF \\
      $x_{t+1} = argmin_{x \in \mathcal{K}}[\eta \sum_{\tau =1}^t \tilde{f_{\tau}^T} x + \mathcal{R}(x)]$
    }
    \ENDFOR
    

  \end{algorithmic}
  \caption{\label{alg:seeifrelin} Bandit online linear optimization}
\end{algorithm}
\end{minipage}
}
\end{center}
\begin{algorithm}[H]
  \begin{algorithmic}[1]
    \STATE{
      Predict $y_t = \gamma e_{i,t}$that is, the $i_{t}$-th standard basis vector scaled by 
      $\gamma$
    }
    \STATE{Observe the cost $f_{t}^T y_{t} = f_{t}(i_t)$}

    \IF{some bucket for $i_{t}$ is empty}
    \STATE{Set $j$ to the index of empty bucket}
    \ELSE
    \STATE{Set j uniformly at random from $\{1, ... ,k\}$}
    \ENDIF \\
    
    \STATE{Update the sample $S_{i,j} = \frac{1}{\gamma}f_{t}(i_t)$} \\
    \IF{$t \leq nk$}
    \STATE{Return $\tilde{\mu_t} = 0$}
    \ELSE
    \STATE{
      Return $\tilde{\mu_t}$ defined as :
      $\forall i \in \{1, 2, ... ,n\} $ Set $ \tilde{\mu_t}(i):= \frac{1}{k} \sum_{j=1}^k S_{i,j}$
    }
    \ENDIF
  \end{algorithmic}
  \caption{\label{alg:seeifrelin} SimplexSample($i_t$)}
\end{algorithm}

\begin{algorithm}[H]
  \begin{algorithmic}[1]
    \STATE{
      Let $\{v_1, ..,v_n\}$ and $\{\lambda_1, ... , \lambda_n\}$ be the set of 
      orthogonal eigenvectors and eigenvalues of $\nabla^2 \mathcal{R}(x_t)$ 
    }
    
    \STATE{
      Choose $i_t$ uniformly at random from $\{1, ..., n\}$ and $\epsilon_t \pm 1$
      with probability $\frac{1}{2}$
    }
    
    \STATE{
      Predict $y_t = x_t + \epsilon_t \lambda_{i_{t}}^{-\frac{1}{2}} v_{i_{t}}$ 
    }

    \STATE{
      Observe the cost $f_{t}^Ty_{t}$
    }
    
    \STATE{
      Return $\tilde{f}_t$ defined as :$ \tilde{f}_t = \tilde{\mu}_t + \tilde{g}_t$\\
      Where $\tilde{g}_t := n(f_{t}^Ty_t - \tilde{\mu_t} y_t) \epsilon_{t} \lambda_{i_{t}} v_{i_{t}}$
      
    }
  \end{algorithmic}
  
  \caption{\label{alg:seeifrelin} EllipsoidSample($x_t$, $\tilde{u}_t$)}
\end{algorithm}


%descriptions
Now that we have formally stated the algorithms, let's try and understand
what each one does and the main ideas being employed. 

\textbf{Main Algorithm}
Naturally we start at the top level algorithm \textit{bandit online linear optimization}.
The algorithm has three main parts: calling SIMPLEX-SAMPLE occasionally, calling ELLIPSOID-SAMPLE
occasionally and updating $x_t$ at every round. \\

We see on lines 1-4 some initialization of the various variables involved. 
We take as parameters $\eta, \mathcal{V} self concordant \mathcal{R}$ and a size 
parameter $k$. We also initialize x to be the index to the minimum point in $\mathcal{R}(x)$ 
and the estimate for the mean reward of each hand to be zero. \\

We then loop through all the rounds. At each round we have the choice of exploring alone 
with a $SIMPLEX-SAMPLE$ or exploring and exploiting with an $ELLIPSOID-SAMPLE$. We 
let the proportion of time spent in either of these helper functions be determined by 
the parameter $\eta$, the size of our reservoir $k$ and which stage we are in with respect to
the optimization procedure as denoted by $t$. Specifically we let the proportion of time 
we spend exploring grow smaller as time goes on (time spent in a SIMPLEX-SAMPLE) and the proportion of time spent exploring and exploiting go up in later stages (time spent in ELLIPSOID-SAMPLE).\\

%TODO clarify what the $i_t$ step is doing 

We can see from the algorithm that our estimate of the mean $\tilde{mu}$ of each arm as computed 
using reservoir sampling only gets updated during the SIMPLEXSAMPLE procedure, but only gets 
used in the ELLIPSOIDSAMPLE procedure. This matches our intuition that exploiting means using the
information that we have collected regarding the losses of each arm and exploring involves gathering this statistic. We also see that $\tilde{f_t}$ is not updated when a SIMPLEXSAMPLE is taken,
only when an ELLIPSOIDSAMPLE is taken. \\
%TODO clarify why the f_t is not updated in more detail


%Describe the SIMPLEX-SAMPLE procedure
\textbf{SIMPLEXSAMPLE}
Now we take a look at the SIMPLEXSAMPLE procedure whose job is simply to perform reservoir sampling on all the points in the feasible set with the given reservoir size.It does this when called, which as discussed earlier is a fraction of time $\frac{nk}{t}$ that decreases with time.
SIMPLEXSAMPLE samples a random point $i \in [n]$ uniformly, the actual sampled point (arm) 
is the corresponding vertex $\gamma e_{i_t}$. This vertex is of the scaled n-dimensional simplex
and by assumption it has to be contained inside of $\mathcal{K}$. The loss is immediately received as $f_t(i_t)$. We then do reservoir sampling: if one of the slots in our reservoir for that
point is empty then we put our loss in that slot, otherwise we kick one element out of the reservoir and put the new loss there. The point kicked out is done so at random uniformly. This procedure exactly implements reservoir sampling and guarantees an unbiased estimate of the mean loss
for that coordinate in the limit. We return the entire vector of means to be used in the main algorithm. Every element of that vector is the estimate of the mean of the coordinate denoted by its
index, at the current round t.
 
%Describe the ELLIPSOID-SAMPLE PROCEDURE
\textbf{ELLIPSOID-SAMPLE}
ELLIPSOIDSAMPLE is where we exploit our knowledge of the adversary (loss surface) by using
our estimate of the mean. This is a modification of a similar procedure outlined in \citep{abernethy}. The paper referenced proves that this sampling procedure is unbiased and has low variation with respect to the regularization. We choose a point again $y_t$ but its chosen from the endpoints of the principal axes of the Dikin ellipsoid $W_1(x_t)$ centered at $x_t$. Recall that our $x_t$ was constructed to minimize the FTRL loss $\eta \sum_{\tau=1}^T\tilde{f_t^T} +\mathcal{R}$.
Since the loss minimized was in terms of our estimate of the loss vector, this is where we are
exploiting our experience with the losses of our coordinates. Furthermore, we update this estimate of $\tilde{f_t}$ at the end of ELLIPSOIDSAMPLE to incorporate our knowledge of the mean $\tilde{\mu_t}$ and our actually loss suffered $f_t$, which is done by adding $f_t$ to 
$\eta (f_{t}^Ty_t - \tilde{\mu_{t}^T}y_t)\epsilon \lambda_{it}^{\frac{1}{2}}v_{it}$ .\\

The procedure is actually quite simple, but there is a lot of mathematical machinery needed to
ensure that our sampled point $y_t$ will be in the feasible set and that our ellipsoid sampling
procedure has nice properties that ensure correctness and efficiency. 

\subsubsection{Guarantees}
We now discuss the main theoretical result of \citep{hazan} which consists of providing a bound on regret in terms of the total variation of the cost vectors. Recall that this bound applies
in the problem setting of oblivious adversarial bandits. Particularly we are in the regime where the coordinates (arms) are not  required to be on the simplex which is known as  online bandit convex optimization.

The main bound on regret is as follows: Let Q be an estimated upperbound on Q, suppose now that Algorithm 1 is run using $\eta = min{\sqrt{\frac{log T}{\eta Q}}, \frac{1}{25n}}$

Then we have the following bound on regret:
\begin{equation}
  E[Regret_t] = O(n \sqrt{\mathcal{V}Q log T} + n log^2(T) + n \mathcal{V}log(T))
\end{equation}

The assumed upperbound on $Q$ is just to simplify the exposition and is not required to carry out the proof as is shown in the paper. The main series of arguments made to demonstrate the bound can be listed as a series of Lemmas, which we now go through. We use the numbering provided by \citep{hazan} for convience in referencing the main paper. Our goal is to state the arguments in the 
order presented by \citep{hazan} and provide extra exposition when nessesary. We prefer to provide context and intution for the arguments rather than simply state the technical arguments which
can be seen in full in \citep{hazan}.

\textbf{Lemma 7}
For any $u \in \mathcal{K}$
\begin{equation}
  E[\sum_{t=1}^T f_{t}^T(y_t - u)] \leq E[\sum_{t=1}^T \tilde{f_{t}^T}(x_t - u)] + 2nlog^2(T).
\end{equation}

This Lemma serves to relate the expected regret of Algorithm 1 with the cost vectors of another algorithm that plays just $x_t$ and not the poin $y_t$ as derived from $x_t$. This relationship is created because ultimately we want to bound $\sum_{t-1}^T \tilde{f_t^T}(x_t - u)$ and the above equation links the key quantities. Further we know that the expectations of $\tilde{t_t}$ and $\tilde{y_t}$ are $f_t$ and $x_t$ respectively and which is why their expected costs can be related per round as we do here. The expected number of rounds grows as $O(nklog(T))$ which explains the last term in the expression: $2nlog^2(T)$. We now go on to take advantage of this formulated relationship by employing typical proof techniques used in FTRL type algorithms.

\textbf{Lemma 8}
For any sequence of cost vectors $\{\tilde{f_1}, ...,\tilde{f_t}\} \in \mathbb{R}^n$
The FTRL algorithm with a $\mathcal{V}$-self concordant barrier $\mathcal{R}$ has the following
guarantee: for any $u \in \mathcal{K}$ we have:

\begin{equation}
  \sum_{t=1}^T \tilde{F_{t}^T} (x_t - u) \leq sum_{t=1}^T (x_t - x_{t+1}) + \frac{2}{\eta}\mathcal{V}  log T
\end{equation}

This statement appeals to a commonly used technique of bounding a FTRL type algorithm by how close the succesive values of the $x-t - x_{t+1}$s are. 
Now we attempt to make some formal statements of how the bound is affected by the different types of sampling procedures, namely ELLIPSOIDSMAPLE of ELLIPSOIDSAMPLE.

\textbf{Lemma 9}
Let t be an ELLIPSOIDSAMPLE step. Then we have 
\begin{equation}
  \tilde{f_t^T}(x_t - x{t+1}) \leq 64 \eta n^2 ||f_t - \mu_t ||^2 + 64 \eta n^2 ||\mu_t -\tilde{\mu_t} ||^2 + 2 \mu_t^T (x_t - x_{t+1}) 
\end{equation} 

To understand Lemma 9, we can turn to the equivalent case for SIMPLEXSAMPLE, where the analysis
is simpler and then fill in anything that would be different in the more complex case of an ELLIPSOIDSAMPLE. We know that in a SIMPLEXSAMPLE $\tilde{f_t} = 0$ so then the following must be true: 
$\tilde{f_t^T}(x_t - x_{t+1} = 0 = 2 \mu_t^T (x_t - x_{t+1}))$. Thus for any SIMPLEXSAMPLE step we get $\tilde{f_t^T}(x_t - x_{t+1}) \leq 64 \eta n ||f_t - \mu_t ||^2 2\mu_t^T (x_t - x_{t+1})$ \\

If we then call the set of all ELLIPSOIDSAMPLE steps $T_E$ we can sum over time periods. When we do so we find components that can be bounded by the above inequalities for SIMPLEXSAMPLE which gets us the final form of Lemma 9.

Now we use some facts about our sampling procedures and how they affect the learning guarentees.

\textbf{Lemma 10}
\begin{equation}
  \sum_{t=1}^T ||f_t - \mu_t||^2 \leq Q_T
\end{equation}

This is simply using the properties of the variance of the estimators of the mean reward vector $\tilde{\mu}$ when using resevoir sampling which is known by \citep{vitter}.
Now we finally introduce total variation into our bounds by recognizing the following upperbound exists in the equalities shown so far.

\textbf{Lemma 11}
\begin{equation}
  E[\sum_{t \in T_E} ||\mu_t - \tilde{\mu_t} ||^2] \leq \frac{log T}{k} Q_T
\end{equation}

This step simply upperbounds the  succesive difference of means seen in  Lemma 11 by the total variation. 
We now get into some messier terrain where we set our hyperparameters to optimal values such that the bound becomes more interpretable and tighter.

\textbf{Lemma 12}
\begin{equation}
  \sum_{t=1}^T \mu_t^T (x_t - x_{t+1}) \leq 2 log(Q_T + 1) + 4
\end{equation}

The first such optimal value we insert is for the resevoir size $k$ where we plug in the bounds from lemmas 10 11 12 into the bound from Lemma 4 and use the value $k=log(T)$ to obtain the simpler expression:\\

$\sum_{t=1}^T f_t^T (x_t - x_{t+1} \leq 128 \eta^2Q + 4log(Q_T +1))+ 8$ \\

Next we do something similar with our value of the exploration-exploitation tradeoff paramter $\eta$. 

Assume that the value of $\eta$ is larger than some value $\frac{\log(Q_t +1)}{8 n^2 Q}$ using this assumption (which we are ultimately in control of since $\eta$ is a hyperparameter that we can choose). Using this assumption gives us the following upperbound on total variation  $Q_T + 1 \leq 8 \eta n^2 Q$. \\

Using lemmas 8 and 7 we can now say for any $u \in \mathcal{K}$  : \\

$E[ \sum_{i=1}^T f_t^T (y_t -y) ] \leq 128 n^2Q + \frac{2 \mathcal{V}}{\eta} logT 
+  2 nlog^2(T) + 4log(Q_T +1) + 8 $

Then choosing a suitable value for $\eta$ that respects the assumption we made earlier about its value then recovers the original regret bound.

\begin{equation}
  E[\sum_{t=1}^T f_t^T (y_t - u)] \leq O(n \sqrt{\mathcal{V}Qlog T} + n \mathcal{V} log(T) + nlog^2(T)) . 
\end{equation}


Let Q be an estimated upperbound on Q, suppose now that Algorithm 1 is run with
$\eta = min{\sqrt{\frac{log T}{\eta Q}}, \frac{1}{25n}}$
\begin{equation}
E[Regret_t] = O(n \sqrt{\mathcal{V}Q log T} + n log^2(T) + n \mathcal{V}log(T))
\end{equation}

\subsection{Partial information setting}
As mentioned in the introduction, the authors in \citep{alon} set about exploring how semantic connections in the action set can influence the bandit model. In particular, they argue that similarities between actions can tighten the regret bound of the algorithm. For instance, in the case of website advertising, they hypothesise that if a user clicks on one ad for running shoes, then the user might be susceptible to click on other ads for running shoes. In contrast, if the user is not interested in one pair of running shoes, chances are that the user is not interested in running shoes at all. Thus, there exists a potential for learning something about the category of running shoes. 

By drawing on graph theoretic notions, the authors clervely model these semantic relationships in a directed graph $G_t=(\mathcal{K},D_t)$ where $\mathcal{K}$ is the set of actions and $D_t$ the set of arcs. Let any arc $(i,j)\in D_t$ for $i\not j$ if and only if playing action $i\in\mathcal{K}$ reveals the loss of action $j\in \mathcal{K}$ at time $t$. Recall that the maximum acyclic subgraph in $G$, denoted $\mas(G)$, is the largest subgraph in $G$ with no directed cycles. Then, based on the size of $\mas(G)$, the authors analyse and give regret bounds on a modified version of the Exp3 algorithm.   

\subsection{LP relaxations}
