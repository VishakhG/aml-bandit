\documentclass{beamer}
\usetheme{metropolis}          
\title{Bandit Problems}
\date{\today}
\author{Vishakh Gopu, Frederik Jensen}
\institute{Advanced machine learning}
\begin{document}
\maketitle

%%%%%%%%%%%%%%%%%%%%%%%%%%%%%%%%%%%%%%%%%%%%%%%%%%%%%%% 
% Introduction
%%%%%%%%%%%%%%%%%%%%%%%%%%%%%%%%%%%%%%%%%%%%%%%%%%%%%%% 

\section{Introduction}
\begin{frame}{Overview of papers}   
  Three main problem settings: 
  \begin{enumerate}
  \item
    Restless bandits
    \begin{itemize}
    \item
      bertsimas, Dimitris,Nin ̃o-Mora, Jose ́. 2000. Rest- less Bandits, Linear Programming
      Relaxations, and a
      Primal-Dual Index Heuristic. Operations Research, 48(1), 80–90.
    \end{itemize}
  \item
    Graph structured feedback
    \begin{itemize}
    \item
      Alon, Noga, Cesa-Bianchi, Nicolo`, Gentile, Claudio, Mannor, Shie, Mansour, Yishay,
      Shamir, Ohad. 2014. Nonstochastic Multi-Armed Bandits with Graph- Structured Feedback.
      CoRR, abs/1409.8428.
    \end{itemize}
  \item
    Handling the variance in rewards
    
    \begin{itemize}
    \item
      Hazan, Elad, Kale, Satyen. 2009. Better Algorithms for Benign Bandits. Pages 38–47 of: 
      Proceedings of the Twentieth Annual ACM-SIAM Symposium on Discrete Algorithms. SODA ’09.
      Philadelphia, PA, USA: Society for Industrial and Applied Mathematics.
    \end{itemize}
  \end{enumerate}
\end{frame}

\begin{frame}{Outline}
  \begin{enumerate}
  \item
    A quick look at restless bandits
    
  \item
    Graph structured feedback and benign bandits in more depth
    \begin{itemize}
    \item
      Problem description
    \item
      Selected results
    \end{itemize}
    
  \item
    Interesting connections and relationships 
  \item
    Questions, exploration, future work
    
  \end{enumerate}
\end{frame}

\section{Restless bandits}
%%%%%%%%%%%%%%%%%%%%%%%%%%%%%%%%%%%%%%%%%%%%%%%%%%%%%%% 
% Restless bandits 
%%%%%%%%%%%%%%%%%%%%%%%%%%%%%%%%%%%%%%%%%%%%%%%%%%%%%%% 

\begin{frame}{Quick overview}
  % Talk about the problem setting and brief  high level description 
  % of the result (avoid showing any results)
  % Mention its in the interest of time
\end{frame}

\section{Non stochastic bandits with graph structured feedback}
%%%%%%%%%%%%%%%%%%%%%%%%%%%%%%%%%%%%%%%%%%%%%%%%%%%%%%% 
% Non stochastic bandits with graph structured feedback
%%%%%%%%%%%%%%%%%%%%%%%%%%%%%%%%%%%%%%%%%%%%%%%%%%%%%%% 
\begin{frame}{High level overview}

\end{frame} 

\begin{frame}{Problem formulation}

\end{frame}

\begin{frame}{KEY RESULT 1}
\end{frame}

\begin{frame}{KEY RESULT 2}

\end{frame}
\section{Better algorithms for benign bandits}
%%%%%%%%%%%%%%%%%%%%%%%%%%%%%%%%%%%%%%%%%%%%%%%%%%%%%%% 
% Total variation
%%%%%%%%%%%%%%%%%%%%%%%%%%%%%%%%%%%%%%%%%%%%%%%%%%%%%%% 
\begin{frame}{High level overview}
  % Description, example and why its interesting
\end{frame}
\begin{frame}{Problem description}
  % Describe connections and terminology
\end{frame}
\begin{frame}{Algorithm high level}
  % talk about FTRL + self concordant regularization
  % Introduce resevoir sampling
\end{frame}
\begin{frame}{Resevoir sampling}
  % Description
  % python code (1 liner)
\end{frame}

\begin{frame}{Bound}
  % describe the bound, why its interesting and what it took to get it to work
\end{frame}

\section{Connections and exploration}
\begin{frame}{Connections between papers}
  \begin{enumerate}
  \item
    Letting novel mathematical tools guide the approach
  \item 
    Interpolating between problem settings
  \end{enumerate}
\end{frame}

\begin{frame}{Exploration}
  % talk about the idea we have.
\end{frame}

\section{Conclusion}
\begin{frame}{Open questions, future work}
  % Open questons from hazan, future ideas
\end{frame}

\section{Questions}
\end{document}